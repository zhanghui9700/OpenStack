% Generated by Sphinx.
\def\sphinxdocclass{report}
\documentclass[letterpaper,10pt,english]{sphinxmanual}
\usepackage[utf8]{inputenc}
\DeclareUnicodeCharacter{00A0}{\nobreakspace}
\usepackage[T1]{fontenc}
\usepackage{babel}
\usepackage{times}
\usepackage[Bjarne]{fncychap}
\usepackage{longtable}
\usepackage{sphinx}
\usepackage{multirow}


\title{OpenStackOperationsGuide Documentation}
\date{March 19, 2013}
\release{1.0.1}
\author{TryStack.cn}
\newcommand{\sphinxlogo}{}
\renewcommand{\releasename}{Release}
\makeindex

\makeatletter
\def\PYG@reset{\let\PYG@it=\relax \let\PYG@bf=\relax%
    \let\PYG@ul=\relax \let\PYG@tc=\relax%
    \let\PYG@bc=\relax \let\PYG@ff=\relax}
\def\PYG@tok#1{\csname PYG@tok@#1\endcsname}
\def\PYG@toks#1+{\ifx\relax#1\empty\else%
    \PYG@tok{#1}\expandafter\PYG@toks\fi}
\def\PYG@do#1{\PYG@bc{\PYG@tc{\PYG@ul{%
    \PYG@it{\PYG@bf{\PYG@ff{#1}}}}}}}
\def\PYG#1#2{\PYG@reset\PYG@toks#1+\relax+\PYG@do{#2}}

\expandafter\def\csname PYG@tok@gd\endcsname{\def\PYG@tc##1{\textcolor[rgb]{0.63,0.00,0.00}{##1}}}
\expandafter\def\csname PYG@tok@gu\endcsname{\let\PYG@bf=\textbf\def\PYG@tc##1{\textcolor[rgb]{0.50,0.00,0.50}{##1}}}
\expandafter\def\csname PYG@tok@gt\endcsname{\def\PYG@tc##1{\textcolor[rgb]{0.00,0.27,0.87}{##1}}}
\expandafter\def\csname PYG@tok@gs\endcsname{\let\PYG@bf=\textbf}
\expandafter\def\csname PYG@tok@gr\endcsname{\def\PYG@tc##1{\textcolor[rgb]{1.00,0.00,0.00}{##1}}}
\expandafter\def\csname PYG@tok@cm\endcsname{\let\PYG@it=\textit\def\PYG@tc##1{\textcolor[rgb]{0.25,0.50,0.56}{##1}}}
\expandafter\def\csname PYG@tok@vg\endcsname{\def\PYG@tc##1{\textcolor[rgb]{0.73,0.38,0.84}{##1}}}
\expandafter\def\csname PYG@tok@m\endcsname{\def\PYG@tc##1{\textcolor[rgb]{0.13,0.50,0.31}{##1}}}
\expandafter\def\csname PYG@tok@mh\endcsname{\def\PYG@tc##1{\textcolor[rgb]{0.13,0.50,0.31}{##1}}}
\expandafter\def\csname PYG@tok@cs\endcsname{\def\PYG@tc##1{\textcolor[rgb]{0.25,0.50,0.56}{##1}}\def\PYG@bc##1{\setlength{\fboxsep}{0pt}\colorbox[rgb]{1.00,0.94,0.94}{\strut ##1}}}
\expandafter\def\csname PYG@tok@ge\endcsname{\let\PYG@it=\textit}
\expandafter\def\csname PYG@tok@vc\endcsname{\def\PYG@tc##1{\textcolor[rgb]{0.73,0.38,0.84}{##1}}}
\expandafter\def\csname PYG@tok@il\endcsname{\def\PYG@tc##1{\textcolor[rgb]{0.13,0.50,0.31}{##1}}}
\expandafter\def\csname PYG@tok@go\endcsname{\def\PYG@tc##1{\textcolor[rgb]{0.20,0.20,0.20}{##1}}}
\expandafter\def\csname PYG@tok@cp\endcsname{\def\PYG@tc##1{\textcolor[rgb]{0.00,0.44,0.13}{##1}}}
\expandafter\def\csname PYG@tok@gi\endcsname{\def\PYG@tc##1{\textcolor[rgb]{0.00,0.63,0.00}{##1}}}
\expandafter\def\csname PYG@tok@gh\endcsname{\let\PYG@bf=\textbf\def\PYG@tc##1{\textcolor[rgb]{0.00,0.00,0.50}{##1}}}
\expandafter\def\csname PYG@tok@ni\endcsname{\let\PYG@bf=\textbf\def\PYG@tc##1{\textcolor[rgb]{0.84,0.33,0.22}{##1}}}
\expandafter\def\csname PYG@tok@nl\endcsname{\let\PYG@bf=\textbf\def\PYG@tc##1{\textcolor[rgb]{0.00,0.13,0.44}{##1}}}
\expandafter\def\csname PYG@tok@nn\endcsname{\let\PYG@bf=\textbf\def\PYG@tc##1{\textcolor[rgb]{0.05,0.52,0.71}{##1}}}
\expandafter\def\csname PYG@tok@no\endcsname{\def\PYG@tc##1{\textcolor[rgb]{0.38,0.68,0.84}{##1}}}
\expandafter\def\csname PYG@tok@na\endcsname{\def\PYG@tc##1{\textcolor[rgb]{0.25,0.44,0.63}{##1}}}
\expandafter\def\csname PYG@tok@nb\endcsname{\def\PYG@tc##1{\textcolor[rgb]{0.00,0.44,0.13}{##1}}}
\expandafter\def\csname PYG@tok@nc\endcsname{\let\PYG@bf=\textbf\def\PYG@tc##1{\textcolor[rgb]{0.05,0.52,0.71}{##1}}}
\expandafter\def\csname PYG@tok@nd\endcsname{\let\PYG@bf=\textbf\def\PYG@tc##1{\textcolor[rgb]{0.33,0.33,0.33}{##1}}}
\expandafter\def\csname PYG@tok@ne\endcsname{\def\PYG@tc##1{\textcolor[rgb]{0.00,0.44,0.13}{##1}}}
\expandafter\def\csname PYG@tok@nf\endcsname{\def\PYG@tc##1{\textcolor[rgb]{0.02,0.16,0.49}{##1}}}
\expandafter\def\csname PYG@tok@si\endcsname{\let\PYG@it=\textit\def\PYG@tc##1{\textcolor[rgb]{0.44,0.63,0.82}{##1}}}
\expandafter\def\csname PYG@tok@s2\endcsname{\def\PYG@tc##1{\textcolor[rgb]{0.25,0.44,0.63}{##1}}}
\expandafter\def\csname PYG@tok@vi\endcsname{\def\PYG@tc##1{\textcolor[rgb]{0.73,0.38,0.84}{##1}}}
\expandafter\def\csname PYG@tok@nt\endcsname{\let\PYG@bf=\textbf\def\PYG@tc##1{\textcolor[rgb]{0.02,0.16,0.45}{##1}}}
\expandafter\def\csname PYG@tok@nv\endcsname{\def\PYG@tc##1{\textcolor[rgb]{0.73,0.38,0.84}{##1}}}
\expandafter\def\csname PYG@tok@s1\endcsname{\def\PYG@tc##1{\textcolor[rgb]{0.25,0.44,0.63}{##1}}}
\expandafter\def\csname PYG@tok@gp\endcsname{\let\PYG@bf=\textbf\def\PYG@tc##1{\textcolor[rgb]{0.78,0.36,0.04}{##1}}}
\expandafter\def\csname PYG@tok@sh\endcsname{\def\PYG@tc##1{\textcolor[rgb]{0.25,0.44,0.63}{##1}}}
\expandafter\def\csname PYG@tok@ow\endcsname{\let\PYG@bf=\textbf\def\PYG@tc##1{\textcolor[rgb]{0.00,0.44,0.13}{##1}}}
\expandafter\def\csname PYG@tok@sx\endcsname{\def\PYG@tc##1{\textcolor[rgb]{0.78,0.36,0.04}{##1}}}
\expandafter\def\csname PYG@tok@bp\endcsname{\def\PYG@tc##1{\textcolor[rgb]{0.00,0.44,0.13}{##1}}}
\expandafter\def\csname PYG@tok@c1\endcsname{\let\PYG@it=\textit\def\PYG@tc##1{\textcolor[rgb]{0.25,0.50,0.56}{##1}}}
\expandafter\def\csname PYG@tok@kc\endcsname{\let\PYG@bf=\textbf\def\PYG@tc##1{\textcolor[rgb]{0.00,0.44,0.13}{##1}}}
\expandafter\def\csname PYG@tok@c\endcsname{\let\PYG@it=\textit\def\PYG@tc##1{\textcolor[rgb]{0.25,0.50,0.56}{##1}}}
\expandafter\def\csname PYG@tok@mf\endcsname{\def\PYG@tc##1{\textcolor[rgb]{0.13,0.50,0.31}{##1}}}
\expandafter\def\csname PYG@tok@err\endcsname{\def\PYG@bc##1{\setlength{\fboxsep}{0pt}\fcolorbox[rgb]{1.00,0.00,0.00}{1,1,1}{\strut ##1}}}
\expandafter\def\csname PYG@tok@kd\endcsname{\let\PYG@bf=\textbf\def\PYG@tc##1{\textcolor[rgb]{0.00,0.44,0.13}{##1}}}
\expandafter\def\csname PYG@tok@ss\endcsname{\def\PYG@tc##1{\textcolor[rgb]{0.32,0.47,0.09}{##1}}}
\expandafter\def\csname PYG@tok@sr\endcsname{\def\PYG@tc##1{\textcolor[rgb]{0.14,0.33,0.53}{##1}}}
\expandafter\def\csname PYG@tok@mo\endcsname{\def\PYG@tc##1{\textcolor[rgb]{0.13,0.50,0.31}{##1}}}
\expandafter\def\csname PYG@tok@mi\endcsname{\def\PYG@tc##1{\textcolor[rgb]{0.13,0.50,0.31}{##1}}}
\expandafter\def\csname PYG@tok@kn\endcsname{\let\PYG@bf=\textbf\def\PYG@tc##1{\textcolor[rgb]{0.00,0.44,0.13}{##1}}}
\expandafter\def\csname PYG@tok@o\endcsname{\def\PYG@tc##1{\textcolor[rgb]{0.40,0.40,0.40}{##1}}}
\expandafter\def\csname PYG@tok@kr\endcsname{\let\PYG@bf=\textbf\def\PYG@tc##1{\textcolor[rgb]{0.00,0.44,0.13}{##1}}}
\expandafter\def\csname PYG@tok@s\endcsname{\def\PYG@tc##1{\textcolor[rgb]{0.25,0.44,0.63}{##1}}}
\expandafter\def\csname PYG@tok@kp\endcsname{\def\PYG@tc##1{\textcolor[rgb]{0.00,0.44,0.13}{##1}}}
\expandafter\def\csname PYG@tok@w\endcsname{\def\PYG@tc##1{\textcolor[rgb]{0.73,0.73,0.73}{##1}}}
\expandafter\def\csname PYG@tok@kt\endcsname{\def\PYG@tc##1{\textcolor[rgb]{0.56,0.13,0.00}{##1}}}
\expandafter\def\csname PYG@tok@sc\endcsname{\def\PYG@tc##1{\textcolor[rgb]{0.25,0.44,0.63}{##1}}}
\expandafter\def\csname PYG@tok@sb\endcsname{\def\PYG@tc##1{\textcolor[rgb]{0.25,0.44,0.63}{##1}}}
\expandafter\def\csname PYG@tok@k\endcsname{\let\PYG@bf=\textbf\def\PYG@tc##1{\textcolor[rgb]{0.00,0.44,0.13}{##1}}}
\expandafter\def\csname PYG@tok@se\endcsname{\let\PYG@bf=\textbf\def\PYG@tc##1{\textcolor[rgb]{0.25,0.44,0.63}{##1}}}
\expandafter\def\csname PYG@tok@sd\endcsname{\let\PYG@it=\textit\def\PYG@tc##1{\textcolor[rgb]{0.25,0.44,0.63}{##1}}}

\def\PYGZbs{\char`\\}
\def\PYGZus{\char`\_}
\def\PYGZob{\char`\{}
\def\PYGZcb{\char`\}}
\def\PYGZca{\char`\^}
\def\PYGZam{\char`\&}
\def\PYGZlt{\char`\<}
\def\PYGZgt{\char`\>}
\def\PYGZsh{\char`\#}
\def\PYGZpc{\char`\%}
\def\PYGZdl{\char`\$}
\def\PYGZhy{\char`\-}
\def\PYGZsq{\char`\'}
\def\PYGZdq{\char`\"}
\def\PYGZti{\char`\~}
% for compatibility with earlier versions
\def\PYGZat{@}
\def\PYGZlb{[}
\def\PYGZrb{]}
\makeatother

\begin{document}

\maketitle
\tableofcontents
\phantomsection\label{index::doc}


Contents:


\chapter{介绍}
\label{Introduction/index:id1}\label{Introduction/index::doc}\label{Introduction/index:welcome-to-openstackoperationsguide-s-documentation}
Contents:


\section{我们为什么写这本书}
\label{Introduction/Why We Wrote This Book::doc}\label{Introduction/Why We Wrote This Book:id1}
2013年2月份,多位作者在5天内集合完成了本书的编写。在编写期间他们消耗了大量的咖啡应和德克萨斯奥斯丁市订到最好的外卖。编写本书的原因是能将我们多年的部署和维护OpenStack的经验贡献和分享给其他人。同时在长时间维护OpenStack的关键岗位工作后,我们也希望能有一份文档成为日常管理员的常备手册,用于指导日常的运维工作。本文档也能在部署决策时提供更多技术细节信息。

编写本书出于2个目标。第一:在设计和建立正式OpenStack环境时提供指导。在阅读完本文档后对于需要收集什么信息,如何组织计算,网络,存储资源和相关软件包有更好的理解第二:提供日常系统管理运维工作的指导。

本书是采用一种称为Book Sprint的方式编写的。这是帮助快速编写书籍的一种开发方式。详细信息可以参考Book Sprint网站(\href{http://www.booksprints.net}{http://www.booksprints.net})

为什么选择OpenStack

OpenStack可以建立一个IaaS的云服务(Infrastructure as a Service,基础架构即服务)。如果你打算在现有资源上建立起相应的云服务方案,OpenStack是一个很好的选择。OpenStack使用开源方式建立,并设计可运行于普通的硬件平台上。事实上,现有的很多OpenStack初始验证系统就是搭建在各种组合的服务器和网络环境上,这些服务器和网络环境就是维护人员手头能使用的空闲资源所拼凑起来的。OpenStack也具有高扩展性,通过简单的增加计算和存储资源,你可以在不影响服务的同时逐渐增加云容量。HP和Racksapce等组织已使用基于OpenStack搭建了大规模的公有云服务。

如果你刚接触OpenStack,很快会发现OpenStack不是一个打包软件集合并通过简单安装后就可以运行。相反,OpenStack是一个系统,通过集成一批不同的技术来构建云。这种构建方式提供了最大限度的灵活性,但刚开始使用时也会被众多的配置选项所混淆。

致谢和感激

OpenStack基金会为本书编写赞助了赴奥斯丁市的机票,住宿(包括风暴后一晚有惊无险的断电事件)和可口食物。在这间奥斯丁市Rackspace的办公室内,我们在一周内募集了1万美元用于书籍的编写。本书所有作者来自于OpenStack基金会。你可以通过基金会网站http://openstack.org/join加入OpenStack基金会。

开始编写的第一天我们在白板上贴满了不同颜色的即时贴。我们用这种方式开始将原来含糊混沌的想法转变成了清晰的章节。

在编写过程中,我们融入了积累的经验和在团队成员中间激发出的想法。在编写的间隙,我们不断回顾本书结构并进一步修正。经过努力工作使得本书能呈现在你的面前。

团队成员包括: Diane Fleming - Diane不知疲倦为OpenStack API文档工作,并愿意继续在本项目上提供帮助。

Tom Fifield - Tom曾在粒子物理学实验(如:欧洲强子对撞机ATLAS项目)中学习了关于计算的可测量性。现在他使用OpenStack的生产环境用于支持澳大利亚的公共研究部门。Tom居住于澳大利亚墨尔本市,他在闲暇时参与OpenStack的文档工作。

Anne Gentle - Anne是OpenStack文档协调员,同时也同Open Street Maps小组一起工作。她作为个人志愿者参与了2011年Google Doc Summit。 Anne曾经在FLOSS Manuals’ Adam Hyde的指导下使用Sprints的方式工作过。Anne居住在德克萨斯的奥斯丁市。

Lorin Hochstein - Lorin本来是一位学者,现在已转变成一个软件开发/运维人员。目前他作为云服务首席架构师任职于Nimbis Services。在这家公司他使用OpenStack环境用于部署科学计算相关应用。他从很早的Cactus版就开始接触OpenStack。在就职于Nimbis Services之前,Lorin工作于南加州大学信息科学委员会(USCISI),主要工作是OpenStack上扩展高性能计算。

Adam Hyde - Adam指导了本书Book Sprint的编写方式。Adam是Book Sprint方法论的发明者也是该方法上最有经验的指导者。关于Book Sprint方法请参考http://www.booksprints.net/。Adam也是FLOSS Manuals的创始人。FLOSS Manuals是一个拥有3000名开发者的社区,该社区为免费软件提供免费编写的说明书。Adam还是Booktype项目的创始人和项目经理,Booktype是书籍撰写,编辑,出版(在线和印刷)的一个开源项目。

Jonathan Proulx - Jonathan是麻省理工学员计算机科学和人工智能实验室的高级系统管理员。为了在研究项目中尽可能充分利用计算资源,他很早就开始使用OpenStack。为了更好的学习他开始参与OpenStack文档的贡献和校验工作。

Everett Toews - Everett Toews就职于Rackspace,工作方向是使得OpenStack和Rackspace更易于使用。他具有多种角色,开发人员,OpenStack倡导者和运维人员。他开发web应用,在研讨会上教学,在全球不同地方发表演说也为大学,商业公司部署OpenStack的生产环境。

Joe Topjian - 在Cybera Joe设计并部署了多个云系统。Cybera是加拿大亚伯达省一个非盈利组织提供基础架构以支持企业家和本地研究员。Joe在维护这些云系统的同时也累计了丰富的troubleshooting技巧。

我们感谢所有Rackspace公司热情的员工。Emma Richards(Rackspace客户关系)照顾了我们的午餐订单,同时不介意大堆从墙上掉落带粘性的即时贴。Betsy Hagemeier(热情的总经理助理)帮我们安排了房间并安顿下来。Rackspace的称作“胜利者”的不动产小组响应超快。(译者:Rackspace还有个Real Estate Team,难道炒房的:)) Adam Powell(IT部)提供了每日必须的宽带,同时为屏幕不够的作者提供了额外的第二台显示器。封面设计师Chris Duncan将周五的一封询问邮件(什么时候能完成什么之类的询问)变成了周五的任务完成邮件,完成结果包括源文件和开源的字体。周三晚上我们同奥斯丁OpenStack Meetup group举行了一小时愉快聚会,感谢Rackspace员工 Katie Schmidt组织了这次聚会。没有你们的支持和鼓励,我们将无法完成本书。

我们也得到了来自于欧洲核子研究组织(European Organisation for Nuclear Research, CERN)的帮助和支持。在我们重新审核本书前,Tim Bell提供了本书大纲的的反馈。Sébastien Han慷慨的贡献出他的blog。Oisin Feeley在我们发出帮助要求后立刻通过邮件给出了反馈,帮助我们做了一些修改。

你需要具备的知识

我们假设你有Linux系统管理的经验并能管理多台Linux环境。具体来说,本书所使用的是Ubuntu的Linux发行版。我们也需要你有一些SQL数据库的经验,本书将涉及到安装和维护MySQL数据库,并需要运行部分SQL查询语句。我们还假设你熟悉虚拟化。

你也需要有Linux环境配置网络部分的经验,OpenStack中最复杂的一部分就是网络的配置。你需要熟悉常用的概念如:DHCP,Linux bridges,VLANs和iptables。除此之外,你还需要能找到具有专业网络硬件经验的人有配置交换机,路由等设备,在OpenStack搭建过程中可能需要他们的参与协助。

(译者:提供以下相关人员信息,有的有照片哦本书作者 Diane Fleming:\href{http://www.linkedin.com/profile/view?id=1992535}{http://www.linkedin.com/profile/view?id=1992535} Tom Fifield:\href{http://www.linkedin.com/in/tomfifield}{http://www.linkedin.com/in/tomfifield} Anne Gentle:\href{http://www.linkedin.com/profile/view?id=1977812}{http://www.linkedin.com/profile/view?id=1977812} Lorin Hochstein:\href{http://www.linkedin.com/profile/view?id=18432136}{http://www.linkedin.com/profile/view?id=18432136} Adam Hyde:\href{http://www.linkedin.com/profile/view?id=34809539}{http://www.linkedin.com/profile/view?id=34809539} Jonathan Proulx:\href{http://www.linkedin.com/profile/view?id=166062713}{http://www.linkedin.com/profile/view?id=166062713} Everett Toews:\href{http://www.linkedin.com/profile/view?id=29659690}{http://www.linkedin.com/profile/view?id=29659690} Joe Topjian:\href{http://www.linkedin.com/profile/view?id=11683813}{http://www.linkedin.com/profile/view?id=11683813}

Rackspace员工: Emma Richards: Betsy Hagemeier:\href{http://www.linkedin.com/profile/view?id=40772775}{http://www.linkedin.com/profile/view?id=40772775} Adam Powell:无 Chris Duncan:无 Katie Schmidt:\href{http://www.linkedin.com/profile/view?id=86814521}{http://www.linkedin.com/profile/view?id=86814521}

CERN: Tim Bell:\href{http://www.linkedin.com/profile/view?id=52400128}{http://www.linkedin.com/profile/view?id=52400128} Sébastien Han:无 Oisin Feeley:无 )


\section{如何贡献本书}
\label{Introduction/How to Work on This Book::doc}\label{Introduction/How to Work on This Book:id1}
本书的编写由个人发起,在阅读的同时我们也希望你能为本书做出贡献。OpenStack文档符合编码准则中涉及的迭代开发,bug跟踪,审查和固化。

如果你在本书中发现Bug或希望能为某章节做出贡献,可以通过访问OpenStack Booktype网站(\href{http://openstack.booktype.pro}{http://openstack.booktype.pro}),在注册用户账号后,开始编辑相应的章节。本书电子版可以访问(\href{http://openstack.booktype.pro/openstack-operations-guide/}{http://openstack.booktype.pro/openstack-operations-guide/})。

来自OpenStack核心文档团队成员之一定期发布本书的最新完整版本。我们计划在OpenStack Summits上特设专人负责该项目。

如果你不能立即修改或不能确认是否确实是文档bug,请在OpenStack Manuals(\href{http://bugs.launchpad.net/openstack-manuals}{http://bugs.launchpad.net/openstack-manuals})上提交该bug并在Extra选项中标示名称为''opsguide''的tag。当你发现可以修正或改善的地方,就可以给自己分配工作任务。同时,OpenStack核心文档团队也会根据文档bug的重要程度进行分类。通过文档How To说明(\href{http://wiki.openstack.org/Documentation/HowTo}{http://wiki.openstack.org/Documentation/HowTo})你可以了解到更多关于文档工作流程的相关信息。


\chapter{架构}
\label{Architecture/index::doc}\label{Architecture/index:id1}
Contents:


\section{云控制器设计}
\label{Architecture/Cloud Controller Design::doc}\label{Architecture/Cloud Controller Design:id1}
OpenStack系统可以横向做大规模的扩展,所有服务都能以分布式方式部署。但在本书中为了简化我们决定将一些服务按照集中式方式部署在一个云控制器节点上(更多架构信息,请参见“样例架构”章节)。云控制器是一个单一节点,用于部署数据库,消息队列服务,认证和权限服务,镜像管理服务,用户仪表盘和API endpoints。

云控制器为多节点的OpenStack部署提供了集中式的管理系统。特别是云控制器管理用户认证并通过消息队列发送消息到所有的系统节点。在我们的样例架构中,云控制器具有一组nova-{\color{red}\bfseries{}*}组建。这些组建可以实现:显示云系统全局状态,服务间的通讯(如:用户认证),维护数据库中云系统信息,同所有的计算和存储节点通过消息队列通讯和提供API的访问方式。每种运行在云控制器上的服务可以分开部署在不同的节点上,通过分开部署可以实现扩展性和可用性。

硬件评估

云控制器可以和计算节点配置一样,未来可以根据云类型和容量重新考虑硬件配置。云控制器也可以通过所有或部分的服务放入虚拟机进行管理,比如:消息队列。本书中我们假设所有服务直接运行的云控制节点的物理环境中。
\begin{description}
\item[{为了准确评估服务器配置,决定是否需要使用全部或部分虚拟化,你需要考虑一下因素:}] \leavevmode
你估计有多少实例运行
当前拥有多少物理机用于计算节点
访问计算或存储服务的用户数量
使用你云服务的用户是如何进行操作的,使用REST API还是通过用户仪表盘
用户认证是否通过外部系统(如:LDAP,Active Directory)
单个实例会保持运行多久

\end{description}

考虑因素                                            扩展考虑
同时会有多少实例运行                      相应需要评估的有,数据库的容量,如果许多实例同时请求状态报告需要考虑单个多个云控制器,新实例启动时需要一定的计算能力。
同时会有多少计算节点运行            确保消息队列服务可以处理相应容量的请求
多少用户会使用API方式访问          如果许多用户发出多条请求,确保云控制器的CPU负载可以处理所有请求。
多少用户会使用仪表板方式访问  仪表盘方式会发送会比API方式发送更多的请求,所以当仪表盘方式是用户主要的操作界面则需要添加更多CPU的计算能力。
同时有多少的nova-api服务在运行     使用一个CPU核心对应一个服务的方式评估云控制器。
单个实例会运行多久                               开始和删除一个实例是在计算节点上处理,但云控制器需要在期间处理API查询和调度控制。
用户认证是否在外部实现                     确保云控制器和外部认证系统间的网络连接正常,且云控制器的CPU有能力处理相应的请求。

分布服务

在我们的样例中,相关的服务都是集中安装在一个物理节点上。但在实际应用中较好的建议是将服务分布到不同的物理节点上。以下是一些我们见到过的部署场景和相应的说明。
\begin{quote}

将glance-{\color{red}\bfseries{}*}和swift-proxy运行在一起:对象存储代理服务所需的IO操作较少,所以相应的Glance的镜像服务可以从物理硬件上获益,同时后端对象存储也有较好的连接性。
运行一个集中专用的数据库服务器:使用专用集中的服务器为其他所有服务提供数据库服务。这种简化方式隔离了数据库升级并允许数据库建立从库以实现Failover。
每个服务运行于一个虚拟机中:在一组运行KVM的服务器中集中部署服务。每个服务对应安装在一个虚拟机中(如:nova-scheduler, rabbitmq,数据库等)。在安装时很难知道在实际运行过程中不同的服务承担的负荷请求,通过这种部署方式下可按照实际服务请求负荷给不同虚拟机分配不同资源来调整实际物理资源的使用。
使用外部的均衡负载:这种部署方式采用专用的昂贵均衡负载硬件,在多台不同的物理硬件上运行nova-api和swift-proxy服务,并通过均衡负载硬件将请求分布到不同的硬件环境上。
\end{quote}

最常见的一个选择是是否需要使用虚拟化。有些服务如:nova-compute,swift-proxy和swift-object服务不应运行于虚拟化环境中。但是,云控制相关服务通常都可以运行在虚拟化环境中。运行在虚拟化环境中有一定的性能损失,但是相应可以运行更多的服务。

数据库
大部分的OpenStack集中服务使用数据库保存统计信息(目前也包括nova-compute节点)。不能访问数据库服务就会导致系统报错。所以,我们建议通过建立数据库集群方式防止数据库的单点故障。

消息队列
大部分的OpenStack服务间使用消息队列方式进行通讯。通常,如果消息队列发生故障或无法访问,OpenStack会系统挂起,最后进入只读状态,该状态下的所有信息就是最后收到的消息所保留的状态。所以我们建议消息队列需要建立在集群的架构上,并RabbitMQ内建支持集群部署。

API
所有的公共访问方式,不管是直接的,通过命令行方式或通过基于web的仪表盘方式都是通过API服务实现。详细的API信息可以访问http://api.openstack.org/

你需要在是否兼容Amazon EC2 API和只支持OpenStack API之间进行选择。当同时使用2种API时常见的不兼容问题涉及到镜像和实例的使用。比如:EC2 API使用包含16进制的ID来标示实例,但OpenStack API使用名字和数字标示。类似情况还有,EC2 API更多的依赖于DNS解析的别名和实例进行连接,而OpenStack则使用IP地址。如果OpenStack没有使用正确方式进行配置,通常的情况就是用户无法访问他们的实例。尽管兼容EC2 API方式存在问题,但是方式下能协助用户从Amazon云迁移到OpenStack上。

就像数据库和消息队列服务,能有多于1个API服务的节点总是件好事。常用的HTTP负载均衡技术也可以实现对于nova-api服务的高可用性。

扩展
API说明文档(\href{http://docs.openstack.org/api/api-specs.html}{http://docs.openstack.org/api/api-specs.html})定义了核心组建,功能和媒体类型。客户端软件可以始终依赖于核心API。OpenStack将一直完整支持该API。通过严格遵循核心API,使得客户端可以使用相同的API在多个不同版本上实现基本一致的功能。

OpenStack API具有扩展性。可通过对于API的扩展增加核心API不具有的功能。新功能的引入,如:MIME类型,动作,状态,头,参数和资源,都通过扩展核心API实现。扩展可以在现有基础上增加新功能而不需要等待新版本,同时也允许供应商提供定制化的功能服务。

调度器
将不同配置的虚拟机调度安排运行在不同性能的物理计算节点上是一个极富挑战的问题,通常该问题也是计算机科学中所涉及到的一系列研究课题。目前有多种技术用于处理调度。其中有:按照虚拟机的配置线性测量法,将物理节点容量均衡分段法等。如何解决调度问题的算法已超出本书范围。OpenStack提供了几种不同的调度引擎,通过访问(\href{http://docs.openstack.org/folsom/openstack-compute/admin/content/ch\_scheduling.html}{http://docs.openstack.org/folsom/openstack-compute/admin/content/ch\_scheduling.html})可以获得详细的调度器说明文档。

出于可用性,大规模部署或高频度调度下,你应该考虑选择使用多个nova-scheduler服务。nova-scheduler服务完全使用消息队列通讯,所以不需要使用负载均衡设备。

镜像
OpenStack镜像分类和交付服务由2部分glance-api和glance-registry组成。前一部分是用于提供镜像交付和计算节点下载镜像文件用途。后一部分则维护了镜像文件的相关元数据信息,该部分服务需要数据库支持。
\begin{description}
\item[{glance-api部分是一个抽象层,它支持多种的后端存储方式。目前支持的有:}] \leavevmode
OpenStack对象存储:将镜像文件已对象的方式进行存储
文件系统:使用普通的文件系统存放镜像文件
S3:从Amazon S3服务上获取镜像(该方式下镜像文件为只读,不能写入镜像)
HTTP:从Web服务器上获取镜像(该方式下镜像文件为只读,不能写入镜像)

\end{description}

如果你已有OpenStack对象存储服务,我们建议你使用对象存储服务来存放镜像。对象存储服务具有良好的扩展性。除非你不需要通过OpenStack上传镜像,否则只有文件系统可以作为实用的选择。

仪表盘
OpenStack仪表盘使用的是运行在Apache httpd服务上的Python web应用。所以,你可以和其他web应用一样的维护方式,确保仪表盘应用可以通过网络访问API服务(包括管理endpoints)。

认证和授权
OpenStack中的认证和授权概念和其他常用系统中的一致。用户使用安全凭证(如:用户名和密码)进行认证。用户可以属于一个或多个组(在OpenStack中称作项目或租户)。比如:一个云系统管理员可以列出所有云中运行的实例,而一个普通用户只能看到他所属组的实例。资源配额(比如可以使用的CPU内核数,磁盘空间等)是和项目相关联的。

OpenStack身份服务(Keystone)提供用户认证和用户属性信息。该服务也为其他的OpenStack服务提供认证。可通过文件policy.json配置策略,请参见用户支持章节获得更多如何配置信息。
\begin{description}
\item[{身份服务通过不同的插件支持不同的后端认证方式。后端认证方式范围从单纯存储到外部系统,包括:}] \leavevmode
基于内存的Key-Value存储
SQL数据库
PAM
LDAP

\end{description}

许多部署方案采用SQL数据库的方式作为认证架构集成入OpenStack。LDAP也是一种常用的认证机制。

网络评估
由于云控制器集中了很多不同的服务,所以云控制器相应的网络通讯产生的流量也很多。比如:如果将OpenStack的镜像服务安装在云控制器上,云控制器需要能以可接受的网速传输镜像文件。

另一个例子是,当你选择使用单主机网络,云控制器将成为所有实例的网关。这种情况下,云控制器必须能支持处理所有从Internet进出的网络访问流量。

建议采用高速网卡,如:10gb网卡。你也可以同时使用2张10gb网卡,并将它们绑定在一起。在双网卡绑定下总带宽不是总能达到20gb,不同的数据传输流将使用不同的网卡。比如:当云控制器正在传输2个镜像文件,每个文件将各自使用一张网卡进行传输,这种情况下,充分利用了各自的10gb带宽。


\chapter{Indices and tables}
\label{index:indices-and-tables}\begin{itemize}
\item {} 
\emph{genindex}

\item {} 
\emph{modindex}

\item {} 
\emph{search}

\end{itemize}



\renewcommand{\indexname}{Index}
\printindex
\end{document}
